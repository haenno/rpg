

\section{Aufgabenbeschreibung}

Am Ende der Spieleentwiklung soll als primäres Ziel der Projektarbeit eine Art RPG (RolePlayingGame) mit Fantasysetting stehen, bei dem es zu aller erst um die Fertigstellung des kompletten Spielumfangs und nicht nur einiger Teilbereiche geht. Das Spiel soll also von Anfang bis Ende durchgespielt werden können und alle an das Projekt gestellten Anforderungen erfüllen und die gewünschten Inhalte bieten. 

Im Einzelnen sind im o. g. Zusammenhang folgende Dinge zu nennen. Das Spiel muss mit einem gängigem Browser online spielbar sein, es soll sowol Singelplayer- als auch Multiplayerelemente enthalten und es wird verschiedene Charakterklassen geben, die sich in ihren Fähigkeiten klar von einander unterscheiden. Im Laufe des Spiels werden sich, bei erreichen bestimmter Grenzen in einem für jeden Charackter eigenem Erfahrungspools, diese Fähigkeiten verbessen. Des weiteren wird es vom Projekteam unabhängig vom Endproduckt ebenfalls als primäres Ziel angesehen, den Projektablauf erfolgreich zu gestallten.


Was ist das Ziel der Projektarbeit? Worin bestehen die (wahrscheinlichen) Herausforderungen? (allg. technisch und auch persönlich)

Primär: 
Fertiges, spielbares Programm 
Projektablauf erfolgreich gestalten 
Web-Technologien nutzen
Fokus auf die technische Umsetzung, Gameplay sekundär

Sekundär: 
Lernkurve
Projektmanagement leben
Erfahrungen im Spieldesign 

Herausforderungen (siehe \ref{2021-11-27-projektskizze-2})

\begin{itemize}
    \item techn. Umsetzbark.
    \item Zeit
    \item Know How
    \item Fokusverlust
    \item geeignete Aufgabefelder für eine Aufteilung finden 
\end{itemize}

Am Ende der Spieleentwiklung soll als primäres Ziel der Projektarbeit eine Art RPG (RolePlayingGame) mit Fantasysetting stehen, bei dem es zu aller erst um die Fertigstellung des kompletten Spielumfangs und nicht nur einiger Teilbereiche geht. Das Spiel soll also von Anfang bis Ende durchgespielt werden können und alle an das Projekt gefordertengestellten Anforderungen erfüllen und die gewünsten Inhalte bieten. Im Einzelnen sind im o. g. Zusammenhang folgende Dinge zu nennen. Das Spiel muss mit einem gängigem Browser online spielbar sein, es soll sowol Singelplayer- als auch Multiplayerelemente enthalten und es wird verschiedene Charakterklassen geben, die sich in ihren Fähigkeiten klar von einander unterscheiden. Im Laufe des Spiels werden sich, bei erreichen bestimmter Grenzen in einem für jeden Charackter eigenem Erfahrungspools, diese Fähigkeiten verbessen. Des weiteren wird es vom Projekteam unabhängig vom Endproduckt ebenfalls als primäres Ziel angesehen, den Projektablauf erfolgreich zu gestallten.
Als Sekundäre Ziele, also den primären Zielen klar untergeortnet, sieht die Projektgruppe die Spielerfahrung der einzelnen Spieler. Es ist allerdings angedacht diese im Ramen der Möglichkeiten so intensiv wie möglich zu gestallten, jedoch steht wie oben bereits erwähnt die Umsetzung der Vorgaben im Vordergrund. Außerdem möchte jeder der Teilnehmer an diesem Projekt Erfahrung im Spieldesign sammeln.

Die Herausforderungen in diesem Projekt sind recht vielfältig und sind grob in technisch und persönlich zu gliedern. Zu ersterem ist zu zählen, dass zu Beginn des Projekts nicht klar ist, ob das vom Team ersteinmal theoretisch entwickelte Konzept auch real, technisch umzusetzen ist. Zum einen weil nicht sicher ist ob die gewählte technische Plattform geeiget ist und zum anderen weil nicht klar ist ob das Know-how des Teams reicht um das gewollte in ein fertiges Endproduckt zu gießen. Außerdem könnten Rechte an Grafiken, Musik/Sounds und vielleicht sogar Programmcode zum Problem werden. Die Persönlichen herausforderungen liegen darin, das ganze Projekt zeilich so zu gestalten, dass es neben der hauptberuflichen Tätigkeit des Projektteams, noch genug Zeit für die Umsetzung bleibt. Und Krankheit oder Quarantäne in der aktuellen Pandemie von einzelnen Teammitgliedern könnte den zeitlichen Ramen ebenfalls in Gefahr bringen. Auch muss für jedes Mitglied des Teams ein geeignetes Aufgabenfeld gefunden werden, so dass die individuellen Fähigkeiten sich möglichst ideal ergänzen um z. B. eine Doppelbelastung eines anderen Teammitgliedes zu vermeiden und die zur verfühgung stehende Zeit optimal zu nutzen. 



