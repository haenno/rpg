

\subsection{Vorstellung des Ergebnisses Rico}

Balancing 
Texte und Story: Spieldesign 

Spieldesign: Am Anfang der Entwicklung musste ein geeignettes Setting für das geplante RPG* gefunden werden. Zur Auswahl standen die beiden klassischen, den meisten bekannten Arten von Spieleuniversen zur Auswahl. Das eine wäre futuristisches Scifi und das andere eher mittelalterliches Fantasy. In der Frühen Phase der Entwicklung war beim Spieldesign ursprünglich geplant ein RPG zu entwickeln, dass viele Elemente vom klassischen Pen and Paper Rollenspielen*, wie z.B. Dungeons and Dragons enthält. Man wollte unterschiedliche Charakterklassen, die sich in ihrer Ausrüstung, wie individuelle Waffen und Rüstungen, sowie ihren speziefischen Fähigkeiten klar von einander unterscheiden. Ein Levelsystem, bei dem die einzelnen Charaktere von Abenteuer zu Abenteuer ihre Fähigkeiten verbessern und bessere Ausrüstung in Form von Beutegut finden können, sollte auch enthalten sein. Das Entwicklerteam hat sich dann für das Fantasysetting entschieden, weil man der Ansicht war, dass sich die vorher genannten Eigenschaften damit besser umsetzen lassen würden. Speziel die unterschiedlichen Eigenschaften der Charakterklassen waren bei dieser Entscheidung massgeblich, denn eine Eingrenzung auf die besonderen Fähigkeiten der unterschiedlichen Spielfiguren, wie z. B. einen Tank*, einen Heiler* und einen Damagedealer*, schien im Scifi-Hintergrund nicht so einfach und für den Spieler nicht schlüssig zugestalten.
Die Überlegungen führten somit schon am Anfang recht schnell zu den bereits erwänten unterscheidlichen Spielfiguren, die der Spieler verkörpern kann. Der Tank(, ein aus dem englischen entlehntes Wort für Panzer,) soll Schaden von anderen und sich selbst abhalten können. Der Heiler soll(, wie das Wort schon verrät,) seine Kameraden und sich heilen können und der Damagedealer soll (,wie das englische Wort andeutet) seinen Schaden und den der anderen Spielfiguren erhöhrn. Wie sich dies im Einzelnen gestalltet, darauf wird später noch genauer eingegangen. Es war angedacht, dass die einzelnen Spielfiguren über Lebenspunkte, Magiepunkte und verschiedene Handlungsoptionen zu geben, die sich abhängig von der Charakterklasse, von einander unterscheiden. 
Auch schien klar zu sin, dass die Würfelmechanik, die Pen and Paper Rollenspielen zu Grunde liegt in das Spiel übernommen werden soll.
Das Spiel sollte vom Ablauf her, aus aufeinander aufbauenden Abenteuern bestehen, die die Charaktere gemeinsam erleben und die mit Hilfe von unterschiedlichen Ansätzen gelöst werden können, bestehen.
Es war geplant als Grundlage für die Regeln, denen das Spiel folgen, also wie ist zu bestimmen, wie und in welcher Reihenfolge der Kampf abläuft oder zu welchen Bedingungen genau wird bestimmt, wann z.b. ein Schwerthieb trift oder nicht, dass sollte auf Grundlage der Opensource Licens von dem Rollenspiel Dungeons and Dragons 3.5 basieren.
Nachdem dieser grobe Ramen festgelegt wurde, haben wir entschieden erst einmal einen Prototypen zu entwickeln, der exemplarisch anhand eines Abenteuers, im folgenden nur noch Spielszene oder Szene genannt, zu klären ob der gesetzte Ramen auch so umsetztbar ist. 
Dabei ist recht schnell aufgefallen, dass vor allem die geplanten unterschiede der einzelnen Charakterklassen, in Form von sich unterscheidenden Handlungsmöglichkeiten in der Szene nicht klar abzugrenzen ist. Außerdem war nicht klar, wie es zu handhaben ist, wenn man z. B. vor einer verschlossenen Tür steht, diese zu öffnen ist, und jede Charakterklasse eine andere Möglichkeit hat dieses Hindernis zu überwinden. Wenn auch bei diesem einfachen Beispiel recht einfach zu klären ist, wie die unterschiedlichen Handlungsmöglichkeiten aussehen, nämlich "Tür eintreten", "Schloss knacken" und die Tür mit einem Zauber öffnen, war nicht klar wie genau das ablaufen sollte. Wer darf den in diesem Moment handeln, der der zuerst auf den Button klickt und was geschied nach entsprechend der unterschiedlichen Auswahl der Möglichkeiten? Sollte alles zu unterschiedlichen Ergebnissen führen oder alles zur selben? Dies macht wiederum die Auswahl was zu tun ist völlig unnötig. Falls die Auswahl zum unterschiedlichen Ergebnissen führt, ensteht ein für uns nicht einzuschätzender Aufwand an Storytelling, grafischen Animationen und an Programierarbeit. Außerdem ist die Komplexität der Kampfmechanick, wie sie in dem zu grunde liegenden Regelsystem von D. a. D. 3.5 recht anspruchsvoll und für einen gemütlichen Abend mit seinen Freunden und auf das Spielen von Angesicht zu Angesicht im Real Live ausgelegt. 
Ein Charakter in diesem Regelwerk besteht aus vielen Eigenschaften, die in Zahlenwerten festgehalten werden. Diese sind unter anderem Stärke, Geschicklichkeit, Konstitution, Intelligenz, Weisheit und Charisma und liegen zwischen 1 und 18, wobei 10-11 als Durschscnitt angesehen werden. All diese Werde beienflussen wie er z.B. kämpfen, stehlen oder mit anderen menschen interagieren kann. Dies ist nur eine grobe Zusammenfassung, denn es gibt noch viel mehr was die Regelmechanik beeinflusst und auch große Spieletitel wie z.B. Baldursgate, die ebenfalls auf d.u.d. 3.5 basieren,haben nicht das gesammte Regelwerk übernommen. 
Auf Grund all der oben genannten Unwägbarkeiten und Komplexität haben wir uns an diesem Punkt zu mehreren, im folgenden erklärten, Konzeptänderung entschieden. 
Der gravierenste ist sicher, dass aus dem RPG eine Art Beat em Up mit Fantasyhintergrund und atmosphärischen Texten und Graficken geworden ist. Dabei gibt es zwar noch immer einzelne Level, die man durschpielt, diese sind aber recht linear und bestehen nur aus einem Kampf gegen einen Gegner, den man entweder gewinnt oder verliert. Ganz grob soll es so ablaufen. Am Anfang der Szene wird ein Hintergrundbild mit Texterklärung eingeblendet und nachdem die oder der Spieler bestätigt haben, wechselt die Ansicht. Auf der linken Seite dieser Ansicht sieht man die Charaktere und auf der anderen Seite den Gegner. Je nach Ausgang des Kampfes wird ein anderer Outrotext angezeigt und man springt zurück in die Auswahl für neue Kämpfe. Es wird keine Ausrüstung geben, die die Charaktere finden können, da dies ein zu umfangriches Datenbankmanagement, für das von uns anvisierte Ziel vorraussetzt. Deas Konzept der Möglichkeit, dass ein Charakter siene Eigenschaften verbessert in dem er bei den Kämpfen an Erfahrung gewinnt, besteht weiterhin, wenn auch etwas einfacher als Angedacht. Auch unterscheidet sich jede Charakterklasse von der anderen durch eine andere Spezialfähigkeit. Die Klassen sind ebenfals geblieben nur hat sich die Bezeichnung etwas konketisiert. Im Einzelnen sind das der Kämfer, er hat die Möglichkeit den erhaltenen Schaden zu reduziren. Dann gibt es noch den Priester, er hat die Fähigkeit zu Heilen und dann ist da noch der Magier, der den Schaden erhhöhen kann. Jeder Kampf gibt eine individuelle Summe an Erfahrung, die dem Charakter gutgeschrieben wird, die er dazu benutzen kann die Fähigkeiten seines Charakters zu verbessern. Der Spiler kann mit der erhaltenen Erfaheung die Lebenspunkte oder die Angriffskraft seines Characters steigern. Ein Charakter erhält immer Erfahrung, egal ob der Kampf gewonnen wird oder ferloren geht. Bei einem Sieg ist diese jedoch deutlich höher als bei einer Niederlage. Das hat zur Folge das man einen Level eventuel mehrfach spielen muss um den nächsten Level erfolgreich abschließen zu können. Der endgültige Tod eines Charakters ist nicht vorgesehen. Die Kampfmechanik wird wie folgt aussehen. Der Kampf ist deutlich vereinfacht und wird Rundenbasiert ablaufen, wobei in jeder Runde der Gegner zuerst angreift. Danach wird jeder Charakter die Möglichkeit haben entweder anzugreifen oder seine Spezialfähigkeit einzusetzen oder auszusetzen. Falls der Charakter seine Spezialfähigkeit nutzt, kann er nicht angreifen und umgekehrt. Die Spezialfähigkeit wirkt mehrere Runden nach. Es ist nicht vorgesehen zu testen ob ein Angriff trift oder nicht, ein Angriff trifft also immer und macht Schaden. Der Gegner verfügt über keine spezialfähigkeiten und macht allen Charakteren den selben für ihn speziefischen Schaden. D. h. der Wolf der 20 Punkte Schaden pro Angriff macht und mehreren Charakteren gegenübersteht, fühgt jedem Spieler die 20 Punkte Schaden zu. Der Schaden der Spieler Summiert sich, so dass drei Spieler die jeweils 20 Schaden zufügen, dem Wolf 60 punkte Lebensenerie abziehen. Vor jedem neuen Kampf verfügen die Spieler wieder über ihre vollen Lebenspunkte. 
Storytelling: Auf Grund der Entwicklung war es nötig zu jeder einzelnen Szene eine Story zu schreiben um dem Spieler ein Gefühl zu geben, was gerade passiert und warum.
Dabei war es wichtig sich auf die Beschreibung der dargestellten Szene zu konzentrieren und nicht abzuschweifen, da ein zu langer Text vom Spieler vielleicht nicht gelesen wird oder als störend empfunden wird. Jedoch muss er lang und intensief genug sein, damit sich der Spieler einene Eindruck von dem Geschehen verschaffen und darin eintauchen kann. Schließlich soll das Spiel ja auch eine Geschichte erzählen und dem Spieler auch eine gewisse Spielerfahrung und im Idealfall einen wiederspielwert geben. Beim Entwickeln der einzelnen Szenen ist aufgefallen, das man nicht immer genau sagen kann was zuerst da war, das Huhn oder das Ei, denn der Text und die Grafik stehen in engem zusammenhang und haben sich gegenseitig beeinflusst. Die Beschreibung der Szene stand üblicherweide zuerst und danach wurde die Szenengrafik entwickelt. Aber manchmal war es auch umgekehrt oder es war nötig den Text anzupassen, weil die Animation z. B. von einem Rudel Wölfe schwieriger war als die von einem einzigen riesigen Wolf. So wurde aus dem Rudel Wölfe was die Gegend terorisiert ein einziger großer Warg. Oder im Fall der Szene im Anwesen war bei ursprünglicher Planung ein Geist vorgesehen, aber beim Schreiben der Geschichte wurde daraus ein Vampier, der viel düsterer ist und sinniger passt. Daraus ist zu erkennen, dass bei einer Spieleentwicklung diese beiden Teilbereiche sehr eng zusammen arbeiten sollten. Dem Programmierer z. B. ist es egal, oder wie unser Leadprogrammer sagte "ich bin da total leidenschaftslos", was für eine grafik oder text an entsprechender Stelle im Code eingefügt werden. As ber Texte und Grafik müssen unbedingt Hand in hand gehen und sich ergänzen um eine überzeugende Spielerfahrung zu schaffen. Im Fall des kamfes gegen den Drachen bin ich bewusst von der reinen Beschreibung der Szene abgewichen und habe viel mehr die Motivation des charakters und entwas Hintergrundgeschichte in den Vordergrund gestellt um beim Spieler die Bewehgründe des Charakters in den Fordergrund zu stellen damit er sich nicht wie bei den anderen Szenen in Handlung sondern mehr in den Akteur hinein versetzen kann. Beim Endboss des Spiels sollte auch etwas besonderes im text stehen.
Balancing: Bei der Ausarbeitung des Ballancings 