
\section{Reflektion}

(TODO!): Jeder einen Punkt ausschreiben. 3-4 gute Beispiele, max 1-2 Seiten. Idee: "Umänderung Spielkonzept von Szenenbasiertem Gameplay (Verschiedene Wege, Adventuremäßig, Dialoge, ...) hin zu reinem Kampf-RPG. "

\begin{itemize}

    \item \textbf{Rundenbasierter vs. chaotischer Spielablauf:} Der Ansatz die Entwicklung durch den Einsatz eines rundenbasierten Ablaufs bzw. Kampfsystems deutlich zu vereinfachen, muss wohl nach den Entwicklungen vom 27.12.2021 (\ref{ref-runden-impl}) zumindest angezweifelt werden. Denn der Aufwand, der durch die notwendige Konzeptionierung und Detailplanung entsteht ist nicht zu unterschätzen. Demengegen stünde bei einem chaotischen Spielablauf lediglich das Handling der Events. 

    \item \textbf{Kleinteilige Aufgabenpakete:} In der Entwicklung eine überschaubare Anzahl an kleinteiligen bzw. Teilufgaben vor sich zu haben, empfand Henning als sehr hilfreich. Man hat damit einen Überblick über die Arbeit der nächsten Tage. Bei der Entwicklung der Rundenlogik entstand durch die Kenntniss um die nächsten Rundenschritte hier ein stets guter Überblick. Der Abschluss jeder einzelnen Teil-Aufgabe sorgte weiter laufend für postive Motivation.

    \item \textbf{Lernkurve und Codequalität:} Eigentlich müsste am Ende eines Entwicklungs-Projektes stets noch einmal von vorne anfangen. Das allein um alles zu korrigieren und anzupassen bzw. auf einen gleichen Quaitätsstand zu bringen, was im Projektverlauf bei den Beteiligten an Fähigkeiten und Wissen gelernt wurde.
    
    \item \textbf{Setting:} Rückwirkend betrachtet, scheint die Wahl des Settings, abhängig von Spielmechanismen, die nachher nicht umgesetzt wurden, nicht mehr so entscheidend, wie zu dem Zeitpunkt der Entwicklung. Denn so wie das Spiel jetzt aufgebaut ist, hätte es durchaus auch im Scifi angesiedelt sein können. Es müssten lediglich der optische Auftritt und die Geschichten abgeändert werden. Der technische Unterbau, speziell der des Kampfmechanismus, welcher mitentscheidend für die Änderung war, ist zum gegenwertigen Zeitpunkt sehr universell und einfach gehalten und könnte so ohne Änderungen übernommen werden.

    \item \textbf{Storytelling/Szenenentwicklung:}  

    \item \textbf{Balancing:} Das Balancing ist, wie weiter oben schon beschrieben sehr kleinteilig und variabel einstellbar aufgebaut, was eine feinfühlige Einstellung der Spieleparameter ermöglichen sollte und es funktioniert auch so wie es soll. Nur das Zusammenspiel all dieser Optionen, besonders der Mechanismus, dass der Charakter seine Erfahrung getrennt für HP und AP ausgeben und dadurch eine extreme Schere zwischen den Charakteren entstehen kann, macht das Balancing auch recht schwer. Hier wäre es vielleicht mit einem Ansatz, mit festen Erfahrungspunkten je Gegner und ein starres Levelsystem, bei dem man bei festgelegten Grenzen einen Level aufsteigt und feste HP und AP Zuwächse je Level bekommt, einfacher. Dadurch ließen sich die Gegner in den aufeinander folgenden Leveln viel besser skalieren, würde aber die Variabilität und Individualität der Charaktere stark beschneiden.

\end{itemize}

