
\section{Reflektion}

(TODO!): Jeder einen Punkt ausschreiben. 3-4 gute Beispiele, max 1-2 Seiten. Idee: "Umänderung Spielkonzept von Szenenbasiertem Gameplay (Verschiedene Wege, Adventuremäßig, Dialoge, ...) hin zu reinem Kampf-RPG. "

\begin{itemize}

    \item \textbf{Rundenbasierter vs. chaotischer Spielablauf:} Der Ansatz die Entwicklung durch den Einsatz eines rundenbasierten Ablaufs bzw. Kampfsystems deutlich zu vereinfachen, muss wohl nach den Entwicklungen vom 27.12.2021 (\ref{ref-runden-impl}) zumindest angezweifelt werden. Denn der Aufwand, der durch die notwendige Konzeptionierung und Detailplanung entsteht ist nicht zu unterschätzen. Demengegen stünde bei einem chaotischen Spielablauf lediglich das Handling der Events. 

    \item \textbf{Kleinteilige Aufgabenpakete:} In der Entwicklung eine überschaubare Anzahl an kleinteiligen bzw. Teilufgaben vor sich zu haben, empfand Henning als sehr hilfreich. Man hat damit einen Überblick über die Arbeit der nächsten Tage. Bei der Entwicklung der Rundenlogik entstand durch die Kenntniss um die nächsten Rundenschritte hier ein stets guter Überblick. Der Abschluss jeder einzelnen Teil-Aufgabe sorgte weiter laufend für postive Motivation.

    \item \textbf{Lernkurve und Codequalität:} Eigentlich müsste am Ende eines Entwicklungs-Projektes stets noch einmal von vorne anfangen. Das allein um alles zu korrigieren und anzupassen bzw. auf einen gleichen Quaitätsstand zu bringen, was im Projektverlauf bei den Beteiligten an Fähigkeiten und Wissen gelernt wurde.
    

\end{itemize}

